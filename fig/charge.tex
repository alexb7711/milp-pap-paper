\pgfplotstableread[col sep = comma]{fig/data/qm-charge.csv}\chargedataquin
\pgfplotstableread[col sep = comma]{fig/data/milp-charge.csv}\chargedatamilp

\begin{subfigures}
    % Quinn ----------
    \begin{figure}[htpb]
        \centering
        \begin{tikzpicture}[scale=0.65]
        \pgfplotstablegetcolsof{\chargedataquin}
        \pgfmathtruncatemacro\NumCols{\pgfplotsretval/2-1} 
        \begin{axis}[title={Quin Charges},
                     xlabel={Time [hr]},
                     xmin=0, xmax=24,
                     ylabel={Charge [kwh]}]
            \foreach \i in {0, ..., \NumCols}
            {
                \pgfmathsetmacro{\u}{\fpeval{2*\i} + 0}
                \pgfmathsetmacro{\etas}{\fpeval{2*\i} + 1}
                \addplot +[mark=none] table[x index=\u, y index =\etas]{\chargedataquin};
            }
        \end{axis}
        \end{tikzpicture}
    \caption{Bus charges for the Quin Modified charging schedule. The charging scheme of the Quin charger is more predictable during the working day.}
    \label{subfig:quin-charge}
    \end{figure}
    \hfill
    % MILP ----------
    \begin{figure}[htpb]
        \centering
        \begin{tikzpicture}[scale=0.65]
        \pgfplotstablegetcolsof{\chargedataquin}
        \pgfmathtruncatemacro\NumCols{\pgfplotsretval/2-1} 
        \begin{axis}[title={MILP Charges},
                     xlabel={Time [hr]},
                     xmin=0, xmax=24,
                     ylabel={Charge [kwh]}]
            \foreach \i in {0, ..., \NumCols}
            {
                \pgfmathsetmacro{\u}{\fpeval{2*\i} + 0}
                \pgfmathsetmacro{\etas}{\fpeval{2*\i} + 1}
                \addplot +[mark=none] table[x index=\u, y index =\etas]{\chargedatamilp};
            }
        \end{axis}
        \end{tikzpicture}
    \label{subfig:milp-charge}
    \caption{The bus charges for the MILP PAP charging schedule. The MILP model allows for guarantees of minimum/maximum changes during the working day as well as charges at the end of the day.}
    \label{subfig:milp-charge}
    \end{figure}
\end{subfigures}